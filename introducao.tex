\chapter{Introdução}

Durante a Revolução Industrial ocorrida no século XVIII, um assunto que começou a se destacar e se tornar muito importante no ramo no ambiente dos processos industriais de produção foi o melhor aproveitamento do tempo e dos recursos, reconheceram que somente uma exploração eficiente dos fatores de produção irá garantir o máximo retorno do capital investido. A necessidade de técnicas avançadas de gestão de tarefas nas indústrias é cada vez mais crítica.. 
	O objetivo do planejamento das operações de produção numa fábrica é gerar um movimento coordenado onde a procura é satisfeita a tempo e com baixos custos. O Scheduling tem como objetivo o melhor aproveitamento dos recursos no tempo necessários para executar um conjunto de tarefas \cite{BAKER}, por isso têm sido desenvolvidas modernas técnicas e ferramentas computadorizadas para um uso otimizado dos recursos e do tempo não só a um nível predictivo (programação da produção a curto prazo), como também a nível reativo (capacidade de reagir a oscilações ou falhas, como o mínimo de disrupções no calendário das operações previamente programado).
	Embora esse campo de pesquisa tenha obtido muitos avanços nos últimos anos, o Scheduling contínua a ser um problema complexo pelo fato de não se enquadrar num modelo genérico, pois as suas características e particularidades variam de caso para caso, e em um mesmo caso pode ocorrer uma infinidade de variações.   

\section{Objetivo e Justificativa}

O objetivo geral deste trabalho é propor uma solução para um problema de planejamento de tarefas em um ambiente industrial utilizando Heurística GRASP

\section{Estrutura do Trabalho}

O presente trabalho tem a seguinte estrutura. No capítulo 2, apresenta-se uma revisão bibliográfica referente aos problemas de sequenciamento, incluindo a sua classificação e métodos de solução, em particular descreve-se o método GRASP, na seção \ref{sec:int_seq_tarefas}.
