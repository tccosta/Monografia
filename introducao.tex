\chapter{Introdução}

Durante a Revolução Industrial ocorrida no século XVIII, um assunto que começou a se destacar e se tornar muito importante no ambiente dos processos industriais de produção foi o melhor aproveitamento do tempo e dos recursos, reconheceram que somente uma exploração eficiente dos fatores de produção irá garantir o máximo retorno do capital investido. A necessidade de técnicas avançadas de gestão de tarefas nas indústrias é cada vez mais crítica.

O objetivo do planejamento das operações de produção numa fábrica, por exemplo, é gerar um movimento coordenado onde a procura é satisfeita a tempo e com baixos custos. O Scheduling tem como objetivo o melhor aproveitamento dos recursos no tempo necessários para executar um conjunto de tarefas \cite{BAKER}, por isso têm sido desenvolvidas modernas técnicas e ferramentas computadorizadas para um uso otimizado dos recursos e do tempo. 
%não só a um nível predictivo (programação da produção a curto prazo), como também a nível reativo (capacidade de reagir a oscilações ou falhas, como o mínimo de disrupções no calendário das operações previamente programado).
	
Embora esse campo de pesquisa tenha obtido muitos avanços nos últimos anos, o Scheduling contínua a ser um problema complexo pelo fato de não se enquadrar num modelo genérico, pois as suas características e particularidades variam de caso para caso, e em um mesmo caso pode ocorrer uma infinidade de variações.   

\section{Objetivo e Justificativa}

O objetivo geral deste trabalho é propor uma solução para um problema de planejamento de tarefas em um ambiente industrial utilizando Heurística GRASP

Em um ambiente indústrial, um processo que depende de máquinas para executar uma série de operaçoes pode fazer com que durante a produção muito tempo seja desperdiçado, e  diminuir esse desperdício de tempo e melhorar o aproveitamento dos recursos, e consequentemente maximizar o lucro, é preciso planejar o processamento de cada operação. Esse planejamento não é simples de realizar, e com base nessa complexidade, uma heurísta é proposta para buscar uma solução para o problema de sequenciamento de tarefas em uma indústria.

\section{Estrutura do Trabalho}

O presente trabalho tem a seguinte estrutura. No capítulo \ref{fundamentos}, apresenta-se uma revisão bibliográfica referente aos problemas de sequenciamento de tarefas, incluindo a sua classificação, é mostrado como um problema é modelado e como sua solução pode ser represntada, ainda é descrito os métodos de solução existentes e algumas aplicações para o problema. No capítulo \ref{grasp} é descirto a heurística GRASP, é descrito em detalhes suas principais rotinas, a rotina de construção de uma solução, a rotina de busca local. No capítulo \ref{fab_aco} é descrito de forma resumido como é  processo de fabricação do aço, bem como cada etapa desse procedimento.



