\chapter{O processo de fabricação do aço}\label{fab_aco}

Neste capitulo é descrito de forma resumida como é o processo de fabricação do aço, em especial o ambiente aciaria, e ainda é especificados os seus parametros para o problema de Job-Shop Scheduling

\section{O Processo}
\cite{UFPR} O aço é um produto siderurgico definido como liga metálica composta principalmente de ferro e pequenas quantidades de carbono. O processo siderurgico pode ser dividido em 4 grandes partes, que são descritas abaixo:
\begin {enumerate}
	\item Preparo das materias-primas (coqueira e sintetização).
	\item Produção de gusa (alto-forno).
	\item Produção de aço (aciaria).
	\item conformação mecânica (laminação)
\end{enumerate}

\subsubsection{Preparo das Materias-primas}
As matérias-primas necessárias para a obtenção do aço são as seguintes: minério de ferro e o carvão mineral. Ambos não são encontrados puros na natureza, sendo necessário então um preparo nas máterias-primas de modo a reduzir o consumo de energia e aumentar a eficiencia no processo.

\subsubsection{Coqueira e Sintetização}
A coqueificação ocorre a uma temperatura de 1300ºC em ausência de ar durante um período de 18 horas, onde ocorre a liberação de substâncias voláteis. O produto resultante desta etapa, o coque, é um material poroso com elevada resistência mecânica, alto ponto de fusão e grande quantidade de carbono.

Na sinterização, a preparação do minério de ferro é feita cuidando-se da granulometria, visto que os grãos mais finos são indesejáveis pois diminuem a permeabilidade do ar na combustão, comprometendo a queima. Para solucionar o problema, adicionam-se materiais fundentes (calcário, areia de sílica ou o próprio sínter) aos grão mais finos.
Com a composição correta, estes elementos são levados ao forno onde a mistura é fundida. Em seguida, o material resultante é resfriado e britado até atingir a granulometria desejada (diâmetro médio de 5mm).
O produto final deste processo é denominado de sínter.

\subsubsection{Alto Forno}
Esta parte do processo de fabricação do aço consiste na redução do minério de ferro, utilizando o coque metalúrgico e outros fundentes, que misturados com o minério de ferro são transformados em ferro gusa. A reação ocorre no equipamento denominado Alto Forno, e constitui uma reação exotérmica. Após a reação, o ferro gusa na forma líquida é transportado nos carros-torpedos (vagões revestidos com elemento refratário) para uma estação de dessulfuração, onde são reduzidos os teores de enxofre a níveis aceitáveis. Também são feitas análises da composição química da liga (carbono, silício, manganês, fósforo, enxofre) e a seguir o carro torpedo transporta o ferro gusa para a aciaria, onde será transformado em aço.

\subsubsection{Aciaria}
Na aciaria, o ferro gusa é transformado em aço através da injeção de oxigênio puro sob pressão no banho de gusa líquido, dentro de um convertedor. A reação, constitui na redução da gusa através da combinação dos elementos de liga existentes (silício, manganês) com o oxigênio soprado, o que provoca uma grande elevação na temperatura, atingindo aproximadamente 1700ºC.
Os gases resultantes do processo são queimados logo na saída do equipamento e a os demais resíduos indesejáveis são eliminados pela escória, que fica a superfície do metal.
Após outros ajustes finos na composição do aço, este é transferido para a próxima etapa que constitui o lingotamento contínuo. 

\subsubsection{Laminação}
Posteriormente, os lingotes devem passar pelo processo de laminação, podendo ser a quente ou a frio, onde se transformarão em chapas através da diminuição da área da seção transversal.


\section{O Ambiente Aciaria}
A Aciaria é local dentro de uma usina siderurgica onde existem máquinas e equipamentos voltados para o processo de transformação do ferro-gusa em diferentes tipos de aço. \cite{IBS}

A tranformação do ferro-gusa em aço envolve:
\begin{itemize}
\item A diminuição dos teores de carbono, silício, fósforo, enxofre e nitrogênio a níveis bastante baixos.
\item A adição de sucata ou minério de ferro para ajustar a temperatura do aço bruto.
\item O ajustes dos teores de carbono, manganês, elementos de liga e da temperatura no forno ou na panela de vazamento.
\end{itemize}

O setor de aciaria compreende os convertedores (refino primário) e o refino secundário, em seguida o aço entra na fase de lingotamento. Este trabalho se limita ao sequenciamento e {\it Scheduling} das corridas de aço apenas no setor de aciaria, ou seja, entre os processos de dessulfuração no carro torpedo e o lingotamento contínuo.

O aço ({\it jobs}) chega na aciaria através dos carros torpedos vindo do setor do alto-forno e o destino é o setor de conformação mecânica, mais precisamente o lingotamento contínuo. O algoritmo irá encontrar a melhor seqüência do jobs em cada “máquina” em função do objetivo predefinido, ou seja, a seqüência que minimiza o tempo de fabricação de
todos os aços a serem produzidos.

O ambiente de aciaria usado nesse trabalho é apenas ilustrativo, a finalidade é fornecer o cenário para o sequenciamento da produção, ou seja, os parâmetros necessários para a resolução Problema de Job-Shop Scheduling.

A aciaria proposta é composta pelos seguintes recursos: Convertedores e Refino Secundário

\subsubsection{Convertedores}
O convertedor é um tipo de forno revestido com tijolos refratários que transforma o ferro usa e sucata em aço. Uma lança sopra oxigênio em alta pressão para o interior do forno produzindo reações químicas que separam as impurezas, como os gases e a escória. No convertedor ocorro o processo chamado refino primário.

\begin{table}[H]
\begin{center}
  \begin{tabular}{|c|c|c|}
    \hline
    {\bf Convertedores} & {\bf Capacidade de produção} & {\bf Tempo médio por corrida} \\
    $\quad$ & (toneladas/corrida)        &    (minutos) \\ \hline
    CV-1 & 300 & 42 \\ \hline
    CV-2 & 400 & 40 \\ \hline
    CV-3 & 500 & 38 \\
    \hline
  \end{tabular}
\end{center}
\caption{Capacidade dos convertedores}
\end{table}

\subsubsection{Refino Secundário}
Para alcançar determinadas propriedades, o aço passa por uma etapa chamada refino secundário, onde são feitas correções em sua composição química e sua temperatura, geralmente essas correções são feitas em um forno panela.


\begin{table}[H]
\begin{center}
  \begin{tabular}{|c|c|c|}
    \hline
    {\bf Fornos Panelas} & {\bf Capacidade de produção} & {\bf Tempo médio por corrida} \\
    $\quad$ & (toneladas/corrida)        &    (minutos) \\ \hline
    RH & 500 & 12 a 50 \\ \hline
   IRUT & 500 & 12 a 50 \\ \hline
    
  \end{tabular}
\end{center}
\caption{Capacidade dos fornos panela}
\end{table}


%\section{Parâmetros para o  problema de Job-Shop Scheduling}
%Neste seção será definido os parâmetros que envolvem o problema de {\it Job-Shop Scheduling}, ou seja, a descrição das máquinas e dos {\it jobs}, definição de suas rotas e seus tempo de processamento em cada máquina. O parâmetro de otimização  utlizado será a minimização do {\it makespan}.

%Os parâmetros necessários para a resolução do {\it Job-Shop Scheduling} são:
%\begin{itemize}
%\item {Jobs} : Os {\it jobs} nesse problema em particular é o aço a ser produzido. Nas siderurugia, de modo geral, é a capacidade de um convertedor que define a dimensão de um lote de produção em cada ciclo de tratamento de aços em aciaria.

%\item {Recursos} : Descrição das características e quantidades de maquinas no sistema.

%\item{Sequencia das máquinas}: A rota de produção é definida pela sequencia das máquinas no sistema, isto é, a ordem das máquinas nas quais os diversos tipos de aço irão ser processados. Uma caracterísitca do 
%problema de {\it Job-shop Scheduling} é que cada tipo de aço possua sua rota de produção definida.

%\item{Tempo de processamento} : Cada {\it job} (aço) possui um determinado tempo de fabricação predeterminado, esse tempo se dá em funcão do tipo de máquina em que é processado, caracterísitcas físicas e químicas do aço, volume de produção, entre outras restrições.

%\end{itemize}

%A figura \ref{params_prob} abaixo traz uma tabela com os parâmetros que serão utilizados para resolução do problema de {\it Job-Shop Scheduling} em um abiente de aciaria.

%\textcolor{red}{AQUI ENTRA A TABELA COM OS DADOS DO PROBLEMA}.

%%\begin{figure}[h!]
%%\begin{center}
%%  \begin{tabular}{|c|c|c|c|c|}
%%    \hline
%%{\bf Tipos de aço
%%  \end{tabular}
%%\end{center}
%%\caption{Parâmetros para o problema de {\it Job-Shop Scheduling em aciaria}}
%%\label{params_prob}
%%\end{figure}


