\chapter{Modelagem do problema}
Neste capítulo será definido os parâmetros que envolvem o problema de {\it Job-Shop Scheduling}, ou seja, a descrição das máquinas e dos {\it jobs}, definição de suas rotas e seus tempo de processamento em cada máquina. O parâmetro de otimização  utlizado será a minimização do {\it makespan}.

\section{Parâmetros para o  problema de Job-Shop Scheduling}
Os parâmetros necessários para a resolução do {\it Job-Shop Scheduling} são:
\begin{itemize}
\item {Jobs} : Os {\it jobs} nesse problema em particular é o aço a ser produzido. Nas siderurugia, de modo geral, é a capacidade de um convertedor que define a dimensão de um lote de produção em cada ciclo de tratamento de aços em aciaria.

\item {Recursos} : Descrição das características e quantidades de maquinas no sistema.

\item{Sequencia das máquinas}: A rota de produção é definida pela sequencia das máquinas no sistema, isto é, a ordem das máquinas nas quais os diversos tipos de aço irão ser processados. Uma caracterísitca do 
problema de {\it Job-shop Scheduling} é que cada tipo de aço possua sua rota de produção definida.

\item{Tempo de processamento} : Cada {\it job} (aço) possui um determinado tempo de fabricação predeterminado, esse tempo se dá em funcão do tipo de máquina em que é processado, caracterísitcas físicas e químicas do aço, volume de produção, entre outras restrições.

\end{itemize}

Abaixo temos a tabela \ref{tipo_aco} que nos mostra as informações mais relevantes dos tipos de aço produzidos neste ambiente fictício.

\begin{table}[H]
\begin{center}
\begin{tabular}{|c|c|c|c|c|c|}
\hline
\multicolumn{ 1}{|c|}{\textbf{Tipos de aço}} & \multicolumn{ 3}{c|}{\textbf{Elementos (\%)}} & \multicolumn{ 1}{c|}{\textbf{Produção (t)}} & \multicolumn{ 1}{c|}{\textbf{Nº de corridas}} \\ \cline{ 2- 4}
\multicolumn{ 1}{|c|}{} & \multicolumn{ 1}{c|}{\textbf{c}} & \multicolumn{ 1}{c|}{\textbf{Mn}} & \multicolumn{ 1}{c|}{\textbf{Si}} & \multicolumn{ 1}{c|}{} & \multicolumn{ 1}{c|}{} \\ 
\multicolumn{ 1}{|c|}{} & \multicolumn{ 1}{c|}{} & \multicolumn{ 1}{c|}{} & \multicolumn{ 1}{c|}{} & \multicolumn{ 1}{c|}{} & \multicolumn{ 1}{c|}{} \\ \hline
A-A & 0,89 & 2,74 & 0,45 & 2000 & 7 \\ \hline
A-B & 0,87 & 2,7 & 0,4 & 3100 & 9 \\ \hline
A-C & 0,85 & 2,72 & 0,43 & 5000 & 10 \\ \hline
A-D & 0,95 & 2,76 & 0,47 & 2500 & 8 \\ \hline
A-E & 0,92 & 2,75 & 0,49 & 3500 & 4 \\ \hline
A-F & 0,88 & 2,8 & 0,5 & 4000 & 6 \\ \hline
A-G & 0,83 & 2,78 & 0,39 & 1000 & 8 \\ \hline
A-H & 0,86 & 2,85 & 0,42 & 8000 & 9 \\ \hline
A-I & 0,9 & 2,68 & 0,52 & 5000 & 11 \\ \hline
A-J & 0,84 & 2,65 & 0,55 & 3000 & 7 \\ \hline
\end{tabular}
\end{center}
\caption{Características de produção dos tipos de aço}
\label{tipo_aco}
\end{table}

A tabela \ref{desc_rota} abaixo nos mostra os parâmetros de roterização que serão utilizados para resolução do problema de {\it Job-Shop Scheduling} em um abiente de aciaria.

\begin{table}[htbp]
\begin{center}
\begin{tabular}{|c|c|c|c|c|c|c|}
\hline
\multicolumn{ 1}{|c|}{\textbf{Tipos de aço}} & \multicolumn{ 4}{c|}{\textbf{Rotas de produção}} & \multicolumn{ 2}{c|}{\textbf{Tempo total }} \\ \cline{ 2- 7}
\multicolumn{ 1}{|c|}{} & \multicolumn{ 1}{c|}{\textbf{1º}} & \multicolumn{ 1}{c|}{\textbf{2º}} & \multicolumn{ 1}{c|}{\textbf{3º}} & \multicolumn{ 1}{c|}{\textbf{4º}} & \multicolumn{ 2}{c|}{\textbf{de processamento }} \\ \cline{ 6- 7}
\multicolumn{ 1}{|c|}{} & \multicolumn{ 1}{c|}{} & \multicolumn{ 1}{c|}{} & \multicolumn{ 1}{c|}{} & \multicolumn{ 1}{c|}{} & \textbf{2º} & \textbf{3º} \\ \hline
A-A & CT & CV & RH & MLC & 279 & 200 \\ \hline
A-B & CT & CV & IRUT & MLC & 275 & 256 \\ \hline
A-C & CT & CV & RH & MLC & 348 & 120 \\ \hline
A-D & CT & CV & IRUT & MLC & 384 & 358 \\ \hline
A-E & CT & CV & RH & MLC & 430 & 300 \\ \hline
A-F & CT & CV & IRUT & MLC & 458 & 473 \\ \hline
A-G & CT & CV & RH & MLC & 372 & 116 \\ \hline
A-H & CT & CV & IRUT & MLC & 502 & 516 \\ \hline
A-I & CT & CV & RH & MLC & 463 & 360 \\ \hline
A-J & CT & CV & IRUT & MLC & 361 & 297 \\ \hline
\end{tabular}
\end{center}
\caption{Parâmetros de roterização na produção do aço}
\label{desc_rota}
\end{table}
Legenda:\\
CT – Carro Torpedo \\
IRUT – Forno IRUT \\				
CV – Convertedores \\		
MLC – Máquina de Lingotamento Contínuo \\				
RH – Forno RH						

Para resolução desse problema será considerado somente as rotas 2 e 3 da tabela acima, ou seja, será resolvido o {\it Scheduling} do aço entre os convertedores e os fornos RH e IRUT.







