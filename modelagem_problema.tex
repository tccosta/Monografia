\chapter{Modelagem do problema}
Neste capítulo será definido os parâmetros que envolvem o problema de {\it Job-Shop Scheduling}, ou seja, a descrição das máquinas e dos {\it jobs}, definição de suas rotas e seus tempo de processamento em cada máquina. O parâmetro de otimização  utlizado será a minimização do {\it makespan}.

\section{Parâmetros para o  problema de Job-Shop Scheduling}
Os parâmetros necessários para a resolução do {\it Job-Shop Scheduling} são:
\begin{itemize}
\item {Jobs} : Os {\it jobs} nesse problema em particular é o aço a ser produzido. Nas siderurugia, de modo geral, é a capacidade de um convertedor que define a dimensão de um lote de produção em cada ciclo de tratamento de aços em aciaria.

\item {Recursos} : Descrição das características e quantidades de maquinas no sistema.

\item{Sequencia das máquinas}: A rota de produção é definida pela sequencia das máquinas no sistema, isto é, a ordem das máquinas nas quais os diversos tipos de aço irão ser processados. Uma caracterísitca do 
problema de {\it Job-shop Scheduling} é que cada tipo de aço possua sua rota de produção definida.

\item{Tempo de processamento} : Cada {\it job} (aço) possui um determinado tempo de fabricação predeterminado, esse tempo se dá em funcão do tipo de máquina em que é processado, caracterísitcas físicas e químicas do aço, volume de produção, entre outras restrições.

\end{itemize}

A figura \ref{params_prob} abaixo traz uma tabela com os parâmetros que serão utilizados para resolução do problema de {\it Job-Shop Scheduling} em um abiente de aciaria.

\textcolor{red}{AQUI ENTRA A TABELA COM OS DADOS DO PROBLEMA}.

%\begin{figure}[h!]
%\begin{center}
%  \begin{tabular}{|c|c|c|c|c|}
%    \hline
%{\bf Tipos de aço
%  \end{tabular}
%\end{center}
%\caption{Parâmetros para o problema de {\it Job-Shop Scheduling em aciaria}}
%\label{params_prob}
%\end{figure}


